% Chapter Template

\chapter{Strong coupling in molecular exciton-plasmon Au nanorod array systems} % Main chapter title

\label{Chapter2} % Change X to a consecutive number; for referencing this chapter elsewhere, use \ref{ChapterX}

%----------------------------------------------------------------------------------------
%	SECTION 1
%----------------------------------------------------------------------------------------
\section{Sample preparation}

In order to find the volume of solvent necessary to get a concentration $10^{-5}$ $mol/l$ we have to use the following formula
\begin{eqnarray}
	Solvent Volume = \frac{powder weight (gr)}{molecular weight (gr/moles)}\cdot\frac{1}{10^{-5} }
\end{eqnarray}

and the solution will be expressed in $ml$


The biological and chemical importance of porphyrins is well established in the current research, those materials are going to be studied in order to find further applications in fields related to light-current transduction; for those purposes a wide range of different porphyrins are synthesized .
 as they are involved in processes such as photosynthesis, metabolic redox reactions and oxygen transport.
 
 Self-assembled structures of porphyrines such as J-aggregate and H-aggregate, provide the possibility inter-molecular transport of excitons with a wide range of potential applications. The fingerprint of such molecular aggregations is provided by the uvVis absorption spectrum, that exhibit red-shift and blue-shift respectively in the Q bands.
 
Typically the porphyrin absorption spectrum in uvVis is characterized by an intense peak at values around $400$ and $500$ $nm$ named {\bf Soret band} and three smaller bands named {\bf Q band}.
\section{Strong coupling regime}


In general light can be thought as a superposition of different wavelengths that can be expressed using an integral

\begin{eqnarray}
	\mathbf{E}(\omega, \mathbf(\kappa) ) = \int d\mathbf{k} \frac{ \partial E }{\partial \mathbf{k} }e^{-i\left( \mathbf{k}\cdot\mathbf{x} - \omega(\mathbf{k})t \right) }
\end{eqnarray}


Where $\omega(\mathbf{k})$ is the the dispersion relation, its derivative $d\omega(\mathbf{k})/d\mathbf{k}$ provides the {\bf group velocity} and the number of points in the $\omega$ $\mathbf{k}$ space for each value of $\omega$ provides the {\bf Density of States} of the system.

In the present work strong coupling between surface plasmon on metallic nanostructured surfaces and molecular exciton was investigated. 

The nanostructured surface was constituted by a set of self-standing gold nanorod with a diameter of $70$ $nm$ , the optically active molecule was a type of porphyrin called TMPyP in acid solution
%-----------------------------------




%----------------------------------------------------------------------------------------
%	SECTION 2
%----------------------------------------------------------------------------------------


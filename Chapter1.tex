% Chapter 1

\chapter{Light-matter interaction} % Main chapter title

\section{Introduction}

Light-matter interaction phenomena can be distinguished into to wide different categories: {\bf resonances} and {\bf scattering}. In the first case light photons have specific energies that correspond to the energy different of two different states in the system, a photon is absorbed by the system and a successive decay to a stationary state is responsible for light re-emission.

An example of scattering is reflection: what we see of the objects around us is the light reflected by them to our eyes. Reflection is an {\bf elastic scattering} that means energy is conserve. An example of {\bf inelastic scattering}, in which photon energy is not conserved is {\bf Raman scattering}, a phenomena widely used for matter investigation


\section{Raman Scattering}

A monochromatic light impinging on a sample brings an electric and magnetic field. At first approximation, molecules on the samples act as a receiving and emitting antenna because they can be thought as a dipole generated by the light itself
\begin{eqnarray}
	\mathbf{p} = \alpha \mathbf{E}
\end{eqnarray}


that resonate with the incoming light and with its vibration generate an electromagnetic signal of the some frequency of the impinging one.\\
Raman spectrum is related to the change in polarizzability of the molecules due to thermal vibronic motions of the molecules themselves.\\
If we think about the polarizability $\alpha$ as a function of the intermolecular distance $\alpha(R)$ and we develop this function in Taylor series

\begin{eqnarray}
	\alpha{\mathbf{R}} \approx \alpha{\mathbf{R}_0} + \frac{\alpha}{\mathbf{R}} \left( \mathbf{R} - \mathbf{R}_0 \right)
\end{eqnarray}


Light at visible spectral range has a frequency that resonate with the uppermost electrons in the atomic shells, in molecular and metallic bonds. when an electron absorb a photon it moves to an excited state, the successive decay back to the ground state can take place in several different ways, that can be distinguished into two different categories: radiative (such as photon to phonon decay) and non-radioative.

In a radiative decay energy can be conserved and the out-coming electron from the excited material has the some wavelength as compared to the incoming one. It is the most probable type of material-light {bf resonance} and even the less interesting for material investigation.

On the other hand, when the incoming light has an energy that correspond to the energy difference between the ground state and the excited one, a resonance phenomena is possible, in which the electron absorb the photon and jump to the excited level. Different decay mechanism (both radioactive and non-radioactive) back to the ground state are possible.
{\bf Fluorescence} is a phenomena that take place on a wide range of materials. When a material is excited by a monochromatic incoming light and it emits photons with a continuum range of different wavelength greater than the incoming one, than a fluorescence phenomena is taking place.

Fluorescence is Resonance phenomena followed by a radioactive and non-radioactive decay, while phenomena such as reflection and Raman are classified as scattering phenomena.
Reflection is an elastic scattering, in which the energy of the out-coming electron is not variated, it is the most probable and so the most intense one, while Raman is a non-linear effect due to the change of polarizability of molecules, the out-coming wavelength can be both greater or lesser than the incoming one and it is classified as an inelastic resonance.

. In the case of fluorescence, electrons jump on an excited molecular level, while in the case of Raman there is no resonance phenomena as it does not depends on the wavelength of the incoming light.

Electrons jump to a so-called virtual level and then they decay back to the ground state


\section{Fluorescence}


\label{Chapter1} % For referencing the chapter elsewhere, use \ref{Chapter1} 

%----------------------------------------------------------------------------------------

% Define some commands to keep the formatting separated from the content 
%\newcommand{\keyword}[1]{\textbf{#1}}
%\newcommand{\tabhead}[1]{\textbf{#1}}
%\newcommand{\code}[1]{\texttt{#1}}
%\newcommand{\file}[1]{\texttt{\bfseries#1}}
%\newcommand{\option}[1]{\texttt{\itshape#1}}

%----------------------------------------------------------------------------------------


\section{Raman and Fluorescence set-up}

The set-up used is an home-made fluorescence microscope with a set of visible lasers at different wavelength ($532$ $nm$, $475$ $nm$ and $600$ $nm$)


\section{Angle resolved and trasmittance uvVis set-up}

The spectrophotometer is a \emph{Thorlabs cs200-M}
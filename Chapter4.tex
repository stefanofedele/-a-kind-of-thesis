% Chapter Template

\chapter{TMPyP porphyrin} % Main chapter title


%From the Airy theory
%jAgg_TMPyP_uvVisAbs
%TMPyP_chemical_structure

TMPyP is a particular type of porphyrin that exhibit a typical absorption spectrum like that one reported in \ref{fig:TMPyP_uvVis_abs}. Four peaks are clearly visible in this spectrum (black line) at $425$ $nm$, $530$ $nm$ $610$ $nm$ and $660$ $nm$  respectively; those peaks where explained via the \emph{Gouterman's four orbital model}.


In this model, two energy levels $1e_u$ are generated. 

The first one is the most intense and is known as \emph{Soret} band and it is due to $S_0\rightarrow S_2$ transition ( $\pi$ to $\pi^*$ transition ?). The difference between the Q band and the Soret band are related to the electronic transition from singlet to singlet and from singlet to triplet.


 while the other three less intense at higher wavelengths are called \emph{Q bands} and they come from $S_0$ to $S_1$ transition. In acid environment ($HCl$ at $pH = 1.0$) the external $\pi$ bonds of different TMPyP molecule bound each other, giving rise a molecular arrangement in a "head to tail" configuration known as \emph{J aggregate} that exhibit a read-shift of the four peaks mentioned before; as reported in the red line in fig. \ref{fig:TMPyP_uvVis_abs}


\begin{figure}[ht!]
	\centering %/home/stefano/Documents/thesis/Figures/eagle.jpeg
		\includegraphics[width=90mm]{/home/stefano/Documents/thesis/Figures/jAgg_TMPyP_uvVisAbs.jpg}
	\caption{TMPyP uvVis absorption spectrum in water solution with $HCl$ ($pH = 1.0$ and j-aggregate ) and without $HCl$  }
	\label{fig:TMPyP_uvVis_abs}
\end{figure}

%In fig. \ref{fig:TMPyP_uvVis_abs}

\begin{figure}[ht!]
	\centering %/home/stefano/Documents/thesis/Figures/eagle.jpeg
		\includegraphics[width=90mm]{/home/stefano/Documents/thesis/Figures/TMPyP_chemical_structure.jpg}
	\caption{TMPyP chemical formula}
\end{figure}

\label{Chapter2} % Change X to a consecutive number; for referencing this chapter elsewhere, use \ref{ChapterX}




\begin{figure}[ht!]
	\centering %/home/stefano/Documents/thesis/Figures/eagle.jpeg
		\includegraphics[width=90mm]{/home/stefano/Documents/thesis/Figures/TMPyPuvVisReflectance.jpg}
	\caption{TMPyP uvVis absorption spectrum in water solution with $HCl$ ($pH = 1.0$ and j-aggregate ) and without $HCl$  }
	\label{fig:TMPyP_uvVis_Reflectance}
\end{figure}


In fig. \ref{fig:TMPyP_uvVis_Reflectance} Reflectance spectra at different angles of TMPyP j-aggregate on gold nanorod are reported. A dip splitting, representing a Fano perturbation is changes is shape as function of the angle
%/home/stefano/Documents/thesis/Figures/TMPyPuvVisReflectance.jpg
%----------------------------------------------------------------------------------------
%	SECTION 1
%----------------------------------------------------------------------------------------

TMPyP aggregates as microrods when it's not j-aggregate and as microsquares when it is aggregated

\section{Sample preparation}

In order to find the volume of solvent necessary to get a concentration $10^{-5}$ $mol/l$ we have to use the following formula
\begin{eqnarray}
	Solvent Volume = \frac{powder weight (gr)}{molecular weight (gr/moles)}\cdot\frac{1}{10^{-5} }
\end{eqnarray}

and the solution will be expressed in $ml$


The biological and chemical importance of porphyrins is well established in the current research, those materials are going to be studied in order to find further applications in fields related to light-current transduction; for those purposes a wide range of different porphyrins are synthesized .
 as they are involved in processes such as photosynthesis, metabolic redox reactions and oxygen transport.
 
 Self-assembled structures of porphyrines such as J-aggregate and H-aggregate, provide the possibility inter-molecular transport of excitons with a wide range of potential applications. The fingerprint of such molecular aggregations is provided by the uvVis absorption spectrum, that exhibit red-shift and blue-shift respectively in the Q bands.
 
Typically the porphyrin absorption spectrum in uvVis is characterized by an intense peak at values around $400$ and $500$ $nm$ named {\bf Soret band} and three smaller bands named {\bf Q band}.
\section{Strong coupling regime}


In general light can be thought as a superposition of different wavelengths that can be expressed using an integral

\begin{eqnarray}
	\mathbf{E}(\omega, \mathbf(\kappa) ) = \int d\mathbf{k} \frac{ \partial E }{\partial \mathbf{k} }e^{-i\left( \mathbf{k}\cdot\mathbf{x} - \omega(\mathbf{k})t \right) }
\end{eqnarray}


Where $\omega(\mathbf{k})$ is the the dispersion relation, its derivative $d\omega(\mathbf{k})/d\mathbf{k}$ provides the {\bf group velocity} and the number of points in the $\omega$ $\mathbf{k}$ space for each value of $\omega$ provides the {\bf Density of States} of the system.

In the present work strong coupling between surface plasmon on metallic nanostructured surfaces and molecular exciton was investigated. 

The nanostructured surface was constituted by a set of self-standing gold nanorod with a diameter of $70$ $nm$ , the optically active molecule was a type of porphyrin called TMPyP in acid solution
%-----------------------------------




%----------------------------------------------------------------------------------------
%	SECTION 2
%----------------------------------------------------------------------------------------

\section{Main Section 2}

Sed ullamcorper quam eu nisl interdum at interdum enim egestas. Aliquam placerat justo sed lectus lobortis ut porta nisl porttitor. Vestibulum mi dolor, lacinia molestie gravida at, tempus vitae ligula. Donec eget quam sapien, in viverra eros. Donec pellentesque justo a massa fringilla non vestibulum metus vestibulum. Vestibulum in orci quis felis tempor lacinia. Vivamus ornare ultrices facilisis. Ut hendrerit volutpat vulputate. Morbi condimentum venenatis augue, id porta ipsum vulputate in. Curabitur luctus tempus justo. Vestibulum risus lectus, adipiscing nec condimentum quis, condimentum nec nisl. Aliquam dictum sagittis velit sed iaculis. Morbi tristique augue sit amet nulla pulvinar id facilisis ligula mollis. Nam elit libero, tincidunt ut aliquam at, molestie in quam. Aenean rhoncus vehicula hendrerit.